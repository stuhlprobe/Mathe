\documentclass{article}
\usepackage{graphicx} % Required for inserting images

\title{Mathematik - Zusammenfassung fürs Abitur}
\author{Maximilian Penke}
\date{January 2024}

\begin{document}

    \maketitle

    \begin{abstract}
        Dies ist eine Zusammenfassung für die Inhalte des Berliner Abiturs von 2024 im Fach Mathematik. Dabei ist des in die drei Hauptthemen unterteilt, wobei es jeweils Differenzierungen gibt. Dafür werden die Inhalte der Einzelthemen erklärt, mit der allgemeinen Umsetzungsweise versehen und darauf folgend mit unterschiedlichen Beispielen.
    \end{abstract}

    \begin{enumerate}

        \item Analysis
        \begin{enumerate}
            \item Gleichungen und Gleichungssysteme
            \item Differenzialrechnung
            \item Ableitungsregeln und Ableitungsbegriff
            \item Integralrechnung
        \end{enumerate}

        \item Analytische Geometrie
        \begin{enumerate}
            \item Zwei- bzw.dreidimensionales Koordinatensystem
            \item Vektoren im Anschauungsraum
            \item Affine Geometrie
            \item Metrische Geometrie
        \end{enumerate}

        \item Stochastik
        \begin{enumerate}
            \item Einführung in die Stochastik
            \item Zufallsgrößen und deren Wahrscheinlichkeitsverteilung
            \item Bernoulli-Ketten und Binomialverteilung als spezielle diskrete Wahrscheinlichkeitsverteilung
            \item Normalverteilung als spezielle stetige Wahrscheinlichkeitsverteilung
            \item Bedingte Wahrscheinlichkeiten
            \item Methoden der beurteilenden Statistik
        \end{enumerate}
    \end{enumerate}

\end{document}

    \begin{enumerate}
        \item Ableitungen
        \item Integrierung
        \item Grenzwerte
        \item Definitionslücken
        \item Rekonstruktion
    \end{enumerate}
    \begin{enumerate}
        \item Polynomfunktionen
        \item Gebrochenrationalefunktionen
        \item Wurzelfunktionen
        \item Exponentialfunktionen
        \item Logarithmusfunktionen
        \item Test
    \end{enumerate}




