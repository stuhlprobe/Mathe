\documentclass{article}
\usepackage{graphicx} % Required for inserting images
\usepackage{geometry}

\title{Mathematik - Zusammenfassung fürs Abitur}
\author{Maximilian Penke}
\date{January 2024}
\newgeometry{vmargin={15mm}, hmargin={12mm,17mm}}

\renewcommand*\contentsname{Gliederung}

\begin{document}

\maketitle

\begin{abstract}
    Dies ist eine Zusammenfassung für die Inhalte des Berliner Abiturs von 2024 im Fach Mathematik. Dabei ist des in die drei Hauptthemen unterteilt, wobei es jeweils Differenzierungen gibt. Dafür werden die Inhalte der Einzelthemen erklärt, mit der allgemeinen Umsetzungsweise versehen und darauf folgend mit unterschiedlichen Beispielen.
\end{abstract}

\tableofcontents
\newpage

\section{Analysis}
\subsection{Gleichungen und Gleichungssysteme}
\paragraph{Gleichung}
Eine Gleichung beschreibt das Verhältnis von zwei Termen mit einem sog. Gleichungsoperator ($=$,$<$,$>$,$<=$,$>=$,$\neq$).
Eine Gleichung kann nur aus Zahlen bestehen (1=1; $1<2$; $4\neq5$; usw.),
aber auch aus Variablen (a=b; x=5; $z^2=25$).
\paragraph{Gleichungssysteme}\label{Gleichungssysteme}
Eine Reihe von Gleichungen, die im selben Kontext stimmen, nennt man Gleichungssystem. Die Gleichungen eines Gleichungssystems werden meist
mit römischen Ziffern (I; II; III) denotiert. Beispiel:
\[ I \quad 5a+2b+c=0\]
\[ II \quad -2a+b=5\]
\[ III \quad c-10=0\]

Mit einem solchen Gleichungssystem sind wir in der Lage, die Werte der Variablen herauszufinden. An unserem Beispiel würde das wie folgt aussehen:

\[ III \quad c-10=0 \quad | +10 \]
\[c = 10\]
c=10 in I einsetzen: 
\[ I \quad 5a+2b+10=0 \quad | -2b \quad -10\]
\[ 5a=-2b-10 \quad | \div 5\]
\[a=-\frac{2}{5}b-10\]
$a=-\frac{2}{5}b-10$ in II einsetzen:
\[ II \quad -2(-\frac{2}{5}b-10)+b=5\]
\[ \frac{4}{5}b+20=5 \quad | -20\]
\[ \frac{4}{5}b=-15 \quad |\cdot \frac{5}{4} \]
\[ b=-18.75\]
b=-18.75 und c=10 in I einsetzen:
\[ I \quad 5a+2(-18.75)+10=0 \quad | -10 \]
\[ 5a-37.5=-10 \quad |+37.5 \]
\[ 5a = 27.5 \quad | \div 5\]
\[ a = 5.5\]

Also ist $a=5.5$, $b=-18.75$, $c=10$

Gleichungsysteme können auch zur Rekonstruktion von Funktionen verwendet werden (siehe \ref{Rekonstruktion}). 

\subsection{Differenzialrechnung}
\paragraph{Was ist Differenzialrechnung?}

Um lokale Änderungen/Steigungen von Funktionen zu bestimmen kann man die Differenzialrechnung verwenden.
Man kann mit ihr ebenfalls Steigungsänderungen bestimmen.
\paragraph{Steigerungsbestimmung von Funktionen über Intervalle}

Um die Steigung von Funktionen über Intervalle zu betrachten sind typische Steigungsdreiecke praktisch,
da man sie visuell gut darstellen kann und sie ein Werkzeug sind welches in der Mittelstuffe bereits
Verwendung gefunden hat.

Differenzenquotient:
\[
    m = {\frac {\Delta y} {\Delta x}}
\]

Differenzialquotient:
\[
    m = \lim_{x \to x_{0}} {{f(x_{0}) - f(x)} \over { x_{0} - x}}
\]



\paragraph{}

\subsection{Ableitungsregeln und Ableitungsbegriff}\label{Ableitungen}
\subsection{Integralrechnung}\label{Integralrechnung}

\subsection{Grenzwerte}
\subsection{Definitionslücken}
\subsection{Rekonstruktion}\label{Rekonstruktion}
\paragraph{}
Zur Rekonstruktion einer Funktion werden gewisse Bedingungen vorrausgesetzt, die in ein Gleichungssystem (siehe \ref{Gleichungssysteme}) umgewandelt werden können.
Beispiel: Ein Helikopterflug kann annährend durch eine Parabel beschrieben werden (Zeit auf der x-Achse in Minuten, Höhe auf der Y-Achse in Kilometer). \\
Der Helikopter erreicht seine maximale Flughöhe von 10 nach 20 Minuten. ( $ \rightarrow f(20)=10 \rightarrow f'(20)=0 $ ) \\
Der Helikopter hebt bei t=0 ab. ($ \rightarrow f(0)=0$) \\

\[ I \quad 10=400a+20b+c \]
\[ II \quad 0=40a+b \]
\[ III \quad c=0 \]

c=0 in I einsetzen:

\[I \quad 10=400a+20b \quad |-400a \]
\[ 10-400a=20b \quad | \div 20\]
\[ 0.5-20a=b\]

$b= 0.5-20a $ in II einsetzen:

\[ 0=40a+(0.5-20a)\]
\[ 0=20a+0.5 \quad |-0.5 \quad \div 20\]
\[ a=-\frac{1}{40} \]

$a=-\frac{1}{40}$ in I einsetzen:

\[ I \quad 400(\frac{1}{40})+20b=10 \]
\[ \frac{1}{10}+20b=10 \quad |-0.1 \quad \div 20\]
\[ b=0.495\]

Mit $a=-\frac{1}{40}$, $b=0.495$ und $c=0$, ergibt sich:

\[f(x)=-\frac{1}{40}x^2+0.495x\]

Das Aufstellen der Bedingungen kann auf verschiedenem Wege erfolgen. Auch Integrale und zweite Ableitungen können dabei eine Rolle spielen.
Lies dazu am Besten \ref{Integralrechnung} und \ref{Ableitungen}.

\section{Analytische Geometrie}
\subsection{Zwei- bzw.dreidimensionales Koordinatensystem}
\subsection{Vektoren im Anschauungsraum}
\subsection{Affine Geometrie}
\subsection{Metrische Geometrie}

\section{Stochastik}
\subsection{Einführung in die Stochastik}
\subsection{Zufallsgrößen und deren Wahrscheinlichkeitsverteilung}
\subsection{Bernoulli-Ketten und Binomialverteilung als spezielle diskrete Wahrscheinlichkeitsverteilung}
\subsection{Normalverteilung als spezielle stetige Wahrscheinlichkeitsverteilung}
\subsection{Bedingte Wahrscheinlichkeiten}
\subsection{Methoden der beurteilenden Statistik}


\begin{enumerate}
    \item Polynomfunktionen
    \item Gebrochenrationalefunktionen
    \item Wurzelfunktionen
    \item Exponentialfunktionen
    \item Logarithmusfunktionen
    \item Test
\end{enumerate}



\end{document}
