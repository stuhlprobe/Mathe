\documentclass{article}
\usepackage{graphicx} % Required for inserting images

\title{Mathematik - Zusammenfassung fürs Abitur}
\author{Maximilian Penke}
\date{January 2024}

\renewcommand*\contentsname{Gliederung}

\begin{document}

\maketitle

\begin{abstract}
    Dies ist eine Zusammenfassung für die Inhalte des Berliner Abiturs von 2024 im Fach Mathematik. Dabei ist des in die drei Hauptthemen unterteilt, wobei es jeweils Differenzierungen gibt. Dafür werden die Inhalte der Einzelthemen erklärt, mit der allgemeinen Umsetzungsweise versehen und darauf folgend mit unterschiedlichen Beispielen.
\end{abstract}

\tableofcontents
\newpage

\section{Analysis}
\subsection{Gleichungen und Gleichungssysteme}
\paragraph{Gleichung}
Eine Gleichung beschreibt das Verhältnis von zwei Termen mit einem sog. Gleichungsoperator ($=$,$<$,$>$,$<=$,$>=$,$\neq$). Eine Gleichung kann nur aus Zahlen bestehen (1=1; $1<2$; $4\neq5$; usw.),
aber auch aus Variablen (a=b; x=5; $z^2=25$).
\paragraph{Gleichungssysteme}
Eine Reihe von Gleichungen, die im selben Kontext stimmen, nennt man Gleichungssystem. Die Gleichungen eines Gleichungssystems werden meist
mit römischen Ziffern (I; II; III) denotiert. Beispiel:
\[ I \quad 5a+2b+c=0\]
\[ II \quad -2a+b=5\]
\[ III \quad c-10=0\]

Mit einem solchen Gleichungssystem sind wir in der Lage, die Werte der Variablen herauszufinden.  An unserem Beispiel würde das wie folgt aussehen:

\[ III \quad c-10=0 \quad | +10 \]
\[c = 10\]
c=10 in I einsetzen: 
\[ I \quad 5a+2b+10=0 \quad | -2b \quad -10\]
\[ 5a=-2b-10 \quad | \div 5\]
\[a=-\frac{2}{5}b-10\]
$a=-\frac{2}{5}b-10$ in II einsetzen:
\[ II \quad -2(-\frac{2}{5}b-10)+b=5\]
\[ \frac{4}{5}b+20=5 \quad | -20\]
\[ \frac{4}{5}b=-15 \quad |\cdot \frac{5}{4} \]
\[ b=-18.75\]
b=-18.75 und c=10 in I einsetzen:
\[ I \quad 5a+2(-18.75)+10=0 \quad | -10 \]
\[ 5a-37.5=-10 \quad |+37.5 \]
\[ 5a = 27.5 \quad | \div 5\]
\[ a = 5.5\]

Also ist $a=5.5$, $b=-18.75$, $c=10$

\subsection{Differenzialrechnung}
\paragraph{Was ist Differenzialrechnung?}
Um lokale Änderungen/Steigungen von Funktionen zu bestimmen kann man die Differenzialrechnung verwenden.
Man kann mit ihr ebenfalls Steigungsänderungen bestimmen.
\paragraph{Steigerungsbestimmung von Funktionen über Intervalle}
Um die Steigung von Funktionen über Intervalle zu betrachten sind typische Steigungsdreiecke praktisch,
da man sie visuell gut darstellen kann und sie ein Werkzeug sind welches in der Mittelstuffe bereits
Verwendung gefunden hat.

\bf{Differenzenquotient:}
\[
    m = {\frac {\Delta y} {\Delta x}}
\]

Differenzialquotient:
\[
    m = \lim_{x \to x_{0}} {{f(x_{0}) - f(x)} \over { x_{0} - x}}
\]



\paragraph{}

\subsection{Ableitungsregeln und Ableitungsbegriff}
\subsection{Integralrechnung}

\section{Analytische Geometrie}
\subsection{Zwei- bzw.dreidimensionales Koordinatensystem}
\subsection{Vektoren im Anschauungsraum}
\subsection{Affine Geometrie}
\subsection{Metrische Geometrie}

\section{Stochastik}
\subsection{Einführung in die Stochastik}
\subsection{Zufallsgrößen und deren Wahrscheinlichkeitsverteilung}
\subsection{Bernoulli-Ketten und Binomialverteilung als spezielle diskrete Wahrscheinlichkeitsverteilung}
\subsection{Normalverteilung als spezielle stetige Wahrscheinlichkeitsverteilung}
\subsection{Bedingte Wahrscheinlichkeiten}
\subsection{Methoden der beurteilenden Statistik}


\begin{enumerate}
    \item Ableitungen
    \item Integrierung
    \item Grenzwerte
    \item Definitionslücken
    \item Rekonstruktion
\end{enumerate}
\begin{enumerate}
    \item Polynomfunktionen
    \item Gebrochenrationalefunktionen
    \item Wurzelfunktionen
    \item Exponentialfunktionen
    \item Logarithmusfunktionen
    \item Test
\end{enumerate}



\end{document}
